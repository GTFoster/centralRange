\documentclass[12pt]{article}
\usepackage{geometry}
\geometry{a4paper}
\usepackage[round, sort]{natbib}
\usepackage{graphicx}
\usepackage[T1]{fontenc}
\usepackage[utf8]{inputenc}
\usepackage{textcomp}
\usepackage{gensymb}
\usepackage{amsmath}
\usepackage{amssymb}
\usepackage{authblk}
\usepackage[running]{lineno}
\usepackage{setspace}
\usepackage{longtable}
\usepackage{hyperref}
\usepackage{times}
%\usepackage{colorbox}


\topmargin 0.0cm
\oddsidemargin 0.2cm
\textwidth 16cm
\textheight 21cm
\footskip 1.0cm

\doublespacing

\renewcommand\Authfont{\fontsize{12}{14.4}\selectfont}
\renewcommand\Affilfont{\fontsize{10}{10.8}\itshape}




\title{\normalsize CentralRange }

\author[a,*]{Grant Foster}

\author[a]{Tad A Dallas}

\author[a]{Cleber Ten Caten}


\affil[a]{Department of Biological Sciences, University of South Carolina, Columbia, SC, 29208 }



\renewcommand\Authands{ and }
\date{ \small *Corresponding author: AUTHOREMAIL@mailbox.sc.edu or gmail.com}



%% Potential reviewers
% Names and email addresses of potential reviewers
%
%
%
%
%
%






\begin{document}


\maketitle

\setstretch{1}
\vspace{-1cm}
\noindent \textbf{Running title}:\\


\noindent \textbf{Author contributions}: All authors contributed to manuscript writing. \\


\noindent \textbf{Acknowledgements}: This work has been supported by ... .  \\


\noindent \textbf{Data accessibility}: $R$ code is available on figshare at \\ \texttt{https://doi.org/ }. \\


\noindent \textbf{Keywords}:  \\

\noindent \textbf{Conflict of interest}: The authors have no conflicts of interest to declare.\\




\clearpage

\linenumbers


\noindent {\large TITLE}

\setstretch{1.3}


\subsection*{Abstract}




















\clearpage

\setstretch{1.3}
\subsection*{Introduction}

\paragraph*{}



\paragraph*{}






\paragraph*{}




% thesis
\paragraph*{}













%----------
\subsection*{Methods}

\paragraph*{Data}




\paragraph*{Species Ranges}
  We constructucted a global pollinator metaweb was created by querying Mangel for plant-pollinator networks. This resulted in a total of 65 globally distributed networks, and a total of 1018 individual species. For each species, we queried global occurance records from GBIF using the \texttt{rgbif} packages. Occurance points were first processed using \texttt{coordinateCleaner}; points were removed if they were present outside of the fossil record, occured within 2km of country centroids, were located on ocean, or were marked as an "introduced" individual.

\paragraph{} To avoid overestimating species range through including spurrious or transient records in GBIF, we used a quantiled minimum convex polygon approach. Whithin each continent, we created the smallest minimum convex polygon that contains 95\% of the occurance points within that continent. Range polygons were not created for species-continent combinations with less than 10 occurence points. 


\paragraph{} For environmental space, we performed a principle components analysis of 18 environmental variables across the globe at a X by X resolution. The first two axes explain X\% of variation of global climate, and are primarily explained by Y VARS \ref{supplement}. We plotted each occurence point for a given species in the cartesian plane defined by these two PAC axis, and the repeated



%\paragraph{} Species ranges were calculated through both minimum convex polygon and alpha hull approaches ($\alpha = 20$). Each species range was calculated seperately for each continent according to a 7 continent model (Asia, Europe, Africa, Oceania, North America, Antartica, and South America); total species range was calculated as the total terrestrial area occupied across all continents.





\paragraph*{Reproducibility}
$R$ code and data to reproduce the analyses is provided at \\
\texttt{https://doi.org/}























% -------------------
\subsection*{Results}


\paragraph*{}



\paragraph*{}



\paragraph*{}























% -------------------
\subsection*{Discussion}

\paragraph*{}



\paragraph*{}



\paragraph*{}



\paragraph*{}



\paragraph*{}








\paragraph*{}




















\clearpage

\bibliography{var}
\bibliographystyle{jae}







\clearpage
\subsection*{Tables}


\begin{table}[ht]
\centering
\caption{A table with caption. }
\label{tab:moran}
\begin{tabular}{lllllll}
  \hline
  covariate & t-SNE axis & obs & exp & sd & $p$-value & $z$-score \\
  \hline
  geography       & 1 & 0.02963 & -0.00032 & 0.00014 & \textbf{$<$ 0.0001} & 216.3 \\
                  & 2 & 0.01930 & -0.00032 & 0.00014 & \textbf{$<$ 0.0001} & 141.7 \\
  age structure   & 1 & 0.00043 & -0.00032 & 0.00001 & \textbf{$<$ 0.0001} & 60.5  \\
                  & 2 & 0.00017 & -0.00032 & 0.00001 & \textbf{$<$ 0.0001} & 39.4  \\
  population size & 1 & 0.00002 & -0.00032 & 0.00003 & \textbf{$<$ 0.0001} & 11.7  \\
                  & 2 & 0.00004 & -0.00032 & 0.00003 & \textbf{$<$ 0.0001} & 12.3  \\
  $R_0$           & 1 & 0.00339 & -0.00032 & 0.00003 & \textbf{$<$ 0.0001} & 110.7 \\
                  & 2 & 0.00135 & -0.00032 & 0.00003 & \textbf{$<$ 0.0001} & 49.8  \\

 \hline
\end{tabular}
\end{table}







\clearpage
\subsection*{Figures}

\begin{figure}[h!]
  \begin{center}
    %\includegraphics[width=\textwidth]{Figures/concept.pdf}
    \caption{A figure with caption. }
    \label{fig:concept}
  \end{center}
\end{figure}


























\clearpage

\newcommand{\beginsupplement}{%
        \setcounter{page}{1}
        \setcounter{table}{0}
        \renewcommand{\thetable}{S\arabic{table}}%
        \setcounter{figure}{0}
        \renewcommand{\thefigure}{S\arabic{figure}}%
        }

\section*{Supplementary materials}

\begin{center}
\textbf{Title}:       \\
\textbf{Authors}:     \\
\end{center}

\beginsupplement


\subsubsection*{}


\begin{figure}[h!]
  \begin{center}
    %\includegraphics[width=0.75\textwidth]{Figures/.pdf}
    \caption{Caption }
    \label{fig:label}
  \end{center}
\end{figure}










\end{document}
